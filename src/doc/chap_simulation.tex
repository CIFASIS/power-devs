\chapter{Simulation with PowerDEVS}
In this chapter, we explain the different options offered by PowerDEVS to simulate a model.

\section{The Simulation Procedure}
The Preprocessor Module is in charge of translating the Block Diagram describing the model into C++ code. 
The result of this translation is a header file \verb"model.h" and a makefile \verb"Makefile.include" located both at the folder \verb"powerdevs/build".

Then, the Preprocessor invokes the C++ compiler which generates the executable file called \verb"model" (or \verb"model.exe" under Windows OS). The executable file is located at the folder \verb"powerdevs/output".

The simulation is then performed by invoking the \verb"model" program with additional parameters.

The Preprocessor can be called directly from PowerDEVS main window at \verb"Simulation->Simulate", which generates the code, compiles it and opens the simulation interface.

\section{Simulation from the Interface}
The simulation interface (Fig.\ref{fig:sim_interface}) is a GUI for the \verb"model" executable file generated by the Preprocessor.

It is invoked by clicking on the blue \emph{play} icon, going to \verb"Simulation->Simulate" at the menu, or just pressing the shorcut key \verb"F5". 

\begin{figure}[h]
% \jpgfile{sim_interface}{4}
 \caption{PowerDEVS Simulation Interface}
 \label{fig:sim_interface}
\end{figure} 

Before performing the simulation, the \emph{final time} must be selected. Then, the simulation can be carried on in different ways:
\begin{itemize}
 \item The button \verb"Run Simulation" performs the simulation in the normal way.
 \item The button \verb"Run Timed" simulates the system in a \emph{Real Time} fashion, i.e., synchronizing the simulation with the physical clock. 
 \item The button \verb"Step(s)" performs $N$ simulation steps, where $N$ is the parameter entered at its left. This can be used for debugging purposes.
\end{itemize}
The simulation interface has also the following functions:
\begin{itemize}
 \item The \verb"Stop" button can be used to break a simulation execution.
 \item The \verb"View Log" button opens the log file of the last simulation. This log file informs the CPU time taken by the simulation, as well as the debug information of the different blocks.
 \item The \verb"Exit" button closes the simulation interface.
 \item The \verb"Run N simulations" option permits selecting the number of times a simulation is executed. Multiple executions of a simulation can be useful in presence of random signals or parameters in order to obtain statistics.
 \item The \verb"Illegitimate Check" option permits selecting the maximum allowed number of simulation steps without advancing the simulation time. If this condition is detected, PowerDEVS aborts the simulation as it considers that the model is \emph{illegitimate}.
\end{itemize}

It is important to take into account that the simulation interface does not compile the model. It only runs the last compiled model. 

Thus, if a model is modified, the changes will not be taken into account until the  model is recompiled from PowerDEVS main window by invoking again the Preprocessor with the play button,  with \verb"Simulation->Simulate", or with the shortcut \verb"F5".

However, if a model contains Scilab parameters, their changes at Scilab workspace are taken into account without the need of recompiling.   
 
\section{Simulation from the Command Line}
The simplest way of executing a simulation is making use of the simulation interface, as explained above. However, in certain applications, it could be of interest to run the simulations from the command line.

For that goal, the preprocessor must be invoked (play button,  \verb"Simulation->Simulate", or \verb"F5") in order to get the model compiled.

Then, the simulation can be performed by invoking the executable file \verb"model" (or \verb"model.exe" under Windows OS) from the command line with the following switches:
\begin{verbatim}
  -i     Run intearctive shell
  -tf t  Set final time to t (double format)
  -ti t  Set initial time to t (double format)
  -rt    Synchronize events with real time
  -d     Debug simulation
  -b n   Stop simulation after n steps in the same time (to detect illegitimate models)
\end{verbatim}
The debug option \verb"-d" prints the simulation engine activity performed at each step.

The interactive shell accessed through the \verb"-i" switch allows to advance the simulation in a step by step fashion for debugging purposes. The shell commands are:
\begin{verbatim}
  r: Run model
  p: Show percentage done
  y [0|1] sYnch events with real time
  x n: Run n simulations
  s [n]: Make n steps
  o Stop simulation
  t time: Set final Time
  b n: Stop simulating after n steps in the same time (illegitimal)
  q: quit
\end{verbatim}
Notice that the options coincide with those of the GUI. However, when running from the command line, the debugging information can be directly seen at the console.

\section{Simulation from Scilab}
The Scilab command \verb"simpd(tf)" simulates the last compiled model up to the final time \verb"tf".

